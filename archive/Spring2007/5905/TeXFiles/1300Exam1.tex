%\documentstyle[12pt,amscd,amssymb,bbm,verbatim,epsfig]{article}

\documentclass{article}
\usepackage{amsmath}
\usepackage{amssymb}


\marginparwidth 0pt
\marginparsep 0pt

\textwidth   6.8in
\textheight  9.3 in
\oddsidemargin 5pt
\evensidemargin 5pt

\topmargin -.6in
\headsep .5in

\begin{document}

\begin{centering}
  \Large       \textbf{CALCULUS 1, Math 1300}         \\[02pt]
  \normalsize  \textbf{February 9, 2005}     \\[02pt]
  \Large       \textbf{1st MIDTERM TEST}   \\[10pt]
\end{centering}

\bigskip

\normalsize

\centerline{
  \fbox{\parbox{7in}{
    \vspace{12pt}
    \textbf{\large YOUR NAME:} \hfill
    \vspace{12pt}
  }}
}

\bigskip

\begin{centering}
    \textbf{Circle Your Section}\\
\end{centering}

\bigskip

{\normalsize
\begin{minipage}[t]{3.0in}
    \textbf{001} \textsc{ R.\ Hasenauer         \dotfill  (8am) }
  \\[4pt]
    \textbf{002} \textsc{ S.\ Kang         \dotfill  (9am) }
  \\[4pt]
    \textbf{003} \textsc{ D.\ Ernst         \dotfill  (9am) }
  \\[4pt]
    \textbf{004} \textsc{ A.\ Golembeski         \dotfill  (10am) }  
\end{minipage} \hfill
\begin{minipage}[t]{3.0in}
    \textbf{005} \textsc{ V.\ Radhkakrishnan          \dotfill  (11am) }
  \\[4pt]
    \textbf{007} \textsc{ J.\ Newhall        \dotfill  (12pm) }
  \\[4pt]
    \textbf{008} \textsc{ T.\ Seguin        \dotfill  (2pm) }
  \\[4pt]
    \textbf{009} \textsc{ C.\ Bruns        \dotfill  (2pm) }    
\end{minipage} \hfill \\

\vspace{1cm}

\begin{centering}
  \textsl{After you get the test back},
  if you consider that something was incorrectly graded,     \\
  \textbf{DO NOT WRITE ON YOUR TEST!}                        \\
  As clearly as possible write down your version
  of the story on a clean sheet of paper,                    \\
  attach it to your test, and give it back to your
  instructor for further consideration.                      \\
\end{centering}

\vspace{1cm}

\centerline{ \large
\begin{tabular}{|||c|r||c|||}
\hline\hline\hline
\textbf{problem} & \textbf{points} & \textbf{score}
\\ \hline\hline
\textbf{1}       &  10 pts         & \hspace{12ex}
\\ \hline
\textbf{2}       &  9 pts         &
\\ \hline
\textbf{3}       &  12 pts         &
\\ \hline
\textbf{4}       &  4 pts         &
\\ \hline
\textbf{5}       &  16 pts         &
\\ \hline
\textbf{6}       &  6 pts         &
\\ \hline
\textbf{7}       &  6 pts         &
\\ \hline
\textbf{8}       &  20 pts         &
\\ \hline
\textbf{9}       &  6 pts         &
\\ \hline
\textbf{10}       &  6 pts         &
\\ \hline
\textbf{11}       &  5 pts         &
\\ \hline\hline
\textbf{TOTAL}   & 100 pts         &
\\ \hline\hline\hline
\end{tabular}
}

\vspace{1cm}

\begin{center}
"On my honor, as a University of Colorado at Boulder student, I have neither given nor received unauthorized assistance on this work."
\end{center}

\centerline{
  \fbox{\parbox{5in}{
    \vspace{8pt}
    \textbf{SIGNATURE:} \hfill
    \vspace{12pt}
  }}
}

\pagebreak

\noindent 1.  (2 points each) Match each of the following functions with their corresponding graph.  Note that there are more graphs than functions.

\vspace{.3cm}

\begin{tabular}{l l l}
(a) & $f(x)=\sqrt{1-x^2}$ & Graph: \\ \\
(b) & $\displaystyle{g(x)=\frac{-2}{1-x}}$ & Graph: \\ \\
(c) & $\displaystyle{h(x)=\frac{x^2-4}{x+2}}$ & Graph: \\ \\
(d) & $k(x)=x^3-x$ & Graph: \\ \\
(e) & $p(x)=x^4-x$ & Graph:
\end{tabular}

\pagebreak

\noindent 2.  (3 points each)  Using the graphs below, evaluate each of the following expressions.

\vspace{.3cm}

(a) $f\circ g(1)$

\vspace{.5cm}

(b) $g\circ f(1)$

\vspace{.5cm}

(c) $\displaystyle{\lim_{x\rightarrow -1}g(x)}$

\vspace{.5 cm}

\pagebreak

\noindent 3.  (3 points each) Consider the following function.
\begin{eqnarray*}
f(x)&=&\begin{cases} -x^2+3, & x<2\\
x, & x \geq 2\end{cases}
\end{eqnarray*}

\vspace{.3 cm}

(a) Find $f(2)$.

\vspace{2.5cm}

(b) Find $\displaystyle{\lim_{x \rightarrow 2^+}f(x)}$ if it exists.

\vspace{2.5cm}

(c) Find $\displaystyle{\lim_{x \rightarrow 2^-}f(x)}$ if it exists.

\vspace{2.5cm}

(d) Find $\displaystyle{\lim_{x \rightarrow 2}f(x)}$ if it exists.

\vspace{3 cm}

\noindent 4.  (4 points) Find $\displaystyle{\lim_{x\rightarrow 2^+}\frac{2-x}{\left|x-2\right|}}$.

\vspace{.3cm}

\pagebreak

\noindent 5.  (4 points each) Evaluate the following limits if they exist.

\vspace{.3 cm}

(a)  $\displaystyle{\lim_{x\rightarrow 3} \frac{x-1}{x+2}}$

\vspace{3cm}

(b) $\displaystyle{\lim_{x\rightarrow -3}\frac{x^2+x-6}{x+3}}$

\vspace{3cm}

(c) $\displaystyle{\lim_{x\rightarrow 0}\frac{\sin^2{2x}}{x^2}}$

\vspace{3cm}

(d) $\displaystyle{\lim_{t\rightarrow 0}\frac{\frac{1}{t+3}-\frac{1}{3}}{t}}$

\vspace{3cm}

\noindent 6.  (6 points) Using the limit definition of the derivative, find the derivative of the function $f(x)=\sqrt{2-x}$.

\vspace{5cm}

\pagebreak

\noindent 7. (6 points) Find an equation of the tangent line to the graph of $\displaystyle{f(x)=2x-\frac{1}{x}}$ at $x=1$.

\vspace{5cm}

\noindent 8. (5 points each) Find the derivative of each of the following functions, but do \underline{not} simplify.

\vspace{.3cm}

(a) $\displaystyle{f(x)=\frac{1}{2}x^3+x^2\sqrt{2}-3x+\pi^2}$

\vspace{3cm}

(b) $g(x)=\big(x^2-5\big)\big(4x+2\big)^2$

\vspace{3cm}

(c) $\displaystyle{h(x)=\frac{-2}{(3-x)^5}}$

\vspace{3cm}

(d) $y=\displaystyle{\frac{x^2-2}{x^2+2}}$

\vspace{3cm}

\pagebreak

\noindent 9. (3 points each)  Consider the following function.
$$g(x)=\frac{x^2-x-2}{x^2-2x-3}$$

\vspace{.3cm}

(a) Where does $g$ have discontinuities?

\vspace{3cm}

(b) Where does $g$ have vertical asymptotes?

\vspace{3cm}

\noindent 10. (6 points) A particle moving in a horizontal straight line along the $x$-axis has position function given by $x(t)=-t^2+24t-225$.  Find the particle's location $x(t)$ when its velocity is zero.

\vspace{6cm}

\noindent 11. (5 points) Draw an example of a function $f$ such that $\displaystyle{\lim_{x \rightarrow 1} f(x)=2}$, but $f(1)\neq2$.

\end{document}
 
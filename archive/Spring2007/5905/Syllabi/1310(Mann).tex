% Course syllabus for MATH 1310 section 001 
% Spring 2007
% LaTeX by Allen Mann

\documentclass[11pt]{article}
\usepackage{graphicx}
\usepackage{amssymb}

\textwidth = 6.5 in
\textheight = 9 in
\oddsidemargin = 0.0 in
\evensidemargin = 0.0 in
\topmargin = 0.0 in
\headheight = 0.0 in
\headsep = 0.0 in
\parskip = 0.2in
\parindent = 0.0in

\newtheorem{theorem}{Theorem}
\newtheorem{corollary}[theorem]{Corollary}
\newtheorem{definition}{Definition}

%Allen's Macro's
\newcommand{\AM}{\textsc{a.m.}}
\newcommand{\PM}{\textsc{p.m.}}
\newcommand{\book}[1]{\textit{#1}} 
\newcommand{\url}[1]{\texttt{#1}} 

\title{MATH 1310-001 \\ Calculus 1 with Biological Applications}
\author{Course Syllabus}
\date{Spring 2007}
\begin{document}

\maketitle
%\vspace{-0.5 in}



\begin{tabbing}
\textit{Office Hours:} \quad   \= Allen.Mann@Colorado.EDU   \kill
\textit{Instructor:}		\> Allen Mann \\
\textit{Office Hours:}    	\> TBA \\
\textit{Office:}			\> Mathematics 362 \\
\textit{E-mail:}			\> \url{Allen.Mann@Colorado.EDU} \\
\textit{Web:}			\> \url{http://math.colorado.edu/\~\,$\!$almann/math1310}
\end{tabbing}

\begin{description}
\item[Lecture:] MTWF 11:00--11:50 \AM\ in MUEN D144.

\item[Recitation:] R 11:00--11:50 \AM\ in MUEN D144.

\item[Help Lab:] MTWR 4:00--6:00 \PM\ in Mathematics 170.

\item[Prerequisites:]
Two years of high school algebra, one year of geometry, and one semester of trigonometry; or MATH 1150 Precalculus Mathematics. A solid understanding of trigonometry and especially algebra is necessary. If your background is inadequate, you should take MATH 1150 before attempting this course.

\item[Course Description:]
This course is similar to MATH 1300 Analytic Geometry and Calculus 1, but a greater emphasis is placed on synthesizing the geometric, numerical, and algebraic aspects of each concept and on exploring ``real world'' applications of calculus. In particular, students will learn to program a graphing calculator to model various biological phenomena. Students who complete this course may continue on to MATH 2520 Introduction to Biometry. This course is not designed to prepare students for MATH 2300 Analytic Geometry and Calculus 2. Students who wish to continue on to MATH 2300 should enroll in MATH 1300.

Students with credit in MATH 1310 may not receive credit in MATH 1080, 1081, 1090, 1100, 1300, APPM 1350, or ECON 1088. Approved for arts and sciences core curriculum: quantitative reasoning and mathematical skills.

\item[Textbook: ]
Adler, Frederick R. \book{Modeling the Dynamics of Life: Calculus and Probability for Life Scientists}. 2nd ed. Brooks/Cole, Belmont, 2005.

\item[Calculator: ]
TI-83 Plus, TI-86, or TI-89 Titanium. 

\item[Grading: ]
\begin{tabbing}
aaaaaaaaaaaaaaaaaaaaaaa\quad    \= Programming Projects \quad  \=    \kill
\> Homework  \> 10\%    \\
\> Programming Projects  \> 10\%    \\
\> Exam 1  \> 15\% \\
\> Exam 2  \> 15\% \\
\> Exam 3  \> 15\% \\
\> Final Exam \> 35\% \\ 
\end{tabbing}

\begin{itemize}
	\item \textbf{Homework:} Homework assigned on Monday will be due on Wednesday at the beginning of class; homework assigned on Wednesday will be due on Friday; homework assigned on Friday will be due on Monday. Late homework will not be accepted. Homework should be neat and legible with no ragged edges and multiple pages stapled together.
	
	\item \textbf{Programming Projects:} Tuesday, students will work on a programming project that will be due at the begining of class on Wednesday. Students may work in small groups, but each student must hand in their individual written work. Late programming projects will not be accepted. The same neatness requirements apply as for homework.
	\item \textbf{Exams:} Three midterm exams will be given February 14, March 14, and April 18 from 5:00--6:30 \PM\ \emph{You must be able to attend these exams.} You must bring your own calculator to the exams.
	\item \textbf{Final Exam:} Wednesday, May 9, 4:30--7:00 \PM
\end{itemize}

\item[Students with disabilities: ]
If you qualify for accommodations because of a disability, please submit to me a letter from Disability Services in a timely manner so that your needs may be addressed.  Disability Services determines accommodations based on documented disabilities. (303) 492-8671, Willard 322, \url{http://www.colorado.edu/disabilityservices}.

\item[Religious Obligations: ]
If you have a religious obligation that conflicts with an assignment or exam, please let me know at least two weeks in advance. Please refer to the University's policy on religious obligations at \url{http://www.colorado.edu/policies/fac\underline{\ }relig.html}.

\item[Classroom Behavior: ]
See polices at \\
\url{http://www.colorado.edu/policies/classbehavior.html} and  \\
\url{http://www.colorado.edu/studentaffairs/judicialaffairs/code.html\#student\underline{\ }code}.

\item[Honor Code: ] Please refer to \url{http://www.colorado.edu/policies/honor.html} and \\
\url{http://www.colorado.edu/academics/honorcode}.

\item[Sexual Harassment: ]
Any student, staff, or faculty member who believes s/he has been sexually harassed should contact the Office of Sexual Harassment (OSH) at (303) 492-2127 or the Office of Judicial Affairs at (303) 492-5550. Information about the OSH and the campus resources available to assist individuals who believe they have been sexually harassed can be obtained at \url{http://www.colorado.edu/sexualharassment/}.
\end{description}

\end{document}
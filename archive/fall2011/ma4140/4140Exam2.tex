\documentclass[11pt]{article}

\usepackage{url}
\usepackage{tikz}
\usepackage{fancyhdr}
\usepackage[margin=.7in]{geometry}
\usepackage[hang,flushmargin,symbol*]{footmisc}
\usepackage{amsmath}
\usepackage{todonotes}
\usepackage{amsthm}
\usepackage{amssymb}
\usepackage{mathtools}
\usepackage{enumitem}
\usepackage{graphicx}
\usepackage{array}
\usepackage{color}
\usepackage{tipa} %to get \textpipe to work
\definecolor{darkblue}{rgb}{0, 0, .6}
\definecolor{grey}{rgb}{.7, .7, .7}
\usepackage[breaklinks]{hyperref}
\hypersetup{
	colorlinks=true,
	linkcolor=darkblue,
	anchorcolor=darkblue,
	citecolor=darkblue,
	pagecolor=darkblue,
	urlcolor=darkblue,
	pdftitle={},
	pdfauthor={}
}

\theoremstyle{definition} 
\newtheorem{theorem}{Theorem}
\newtheorem{lemma}[theorem]{Lemma}
\newtheorem{claim}[theorem]{Claim}
\newtheorem{corollary}[theorem]{Corollary}
\newtheorem{conjecture}[theorem]{Conjecture}
\newtheorem{definition}[theorem]{Definition}
\newtheorem{example}[theorem]{Example}
\newtheorem{remark}[theorem]{Remark}
\newtheorem{important}[theorem]{Important Note}
\newtheorem{recall}[theorem]{Recall}
\newtheorem{note}[theorem]{Note}
\newtheorem{question}[theorem]{Question}
\newtheorem*{definition*}{Definition}

\setlength{\parindent}{0pt}
\setlength{\fboxsep}{10pt}

%\newcolumntype{x}[1]{%
%>{\centering\hspace{0pt}}p{#1}}%
%\renewcommand\arraystretch{2}

%%%%%%Header/Footer%%%%%%%

\pagestyle{fancy}

\lhead{\scriptsize  MA4140: Abstract Algebra (Fall 2011)} 
\chead{} 
\rhead{\scriptsize Exam 2} 
\lfoot{\scriptsize This work is licensed under the \href{http://creativecommons.org/licenses/by-sa/3.0/us/}{Creative Commons Attribution-Share Alike 3.0 License}.} 
\cfoot{} 
\rfoot{\scriptsize Written by \href{http://oz.plymouth.edu/~dcernst}{D.C. Ernst}} 
\renewcommand{\headrulewidth}{0.4pt} 
\renewcommand{\footrulewidth}{0.4pt} 

%%%%%%%%%%%%%%%%%%%

\begin{document}

\begin{center}

{\Large\bf MA4140: Abstract Algebra (Fall 2011)}\\
\smallskip
{\Large\bf Exam 2}

\bigskip

  \fbox{\parbox{7in}{
    \vspace{10pt}
    \textbf{\large Your Name:}
    \vspace{10pt}
  }}
  
  \bigskip
  
  \fbox{\parbox{7in}{
    \vspace{10pt}
    \textbf{\large Names of any collaborators:}
    \vspace{10pt}
  }}

\end{center}

\setlength{\fboxsep}{10pt}

\section*{Instructions}

This exam is worth a total of 68 points and 15\% of your overall grade.  For each part of the exam, read the instructions carefully.

\bigskip

I expect your proofs to be \emph{well-written, neat, and organized}.  You should write in \emph{complete sentences}.  Do not turn in rough drafts.  What you turn in should be the ``polished'' version of potentially several drafts.  Feel free to type up your final version.  

\bigskip

The \LaTeX\ source file of this exam is also available if you are interested in typing up your solutions using \LaTeX.  I'll help you do this if you'd like.

\bigskip

The simple rules for the exam are:

\begin{enumerate}
\item You may freely use any theorems that we have discussed in class, but you should make it clear where you are using a previous result and which result you are using.  For example, if a sentence in your proof follows from Theorem 28, then you should say so.
\item Unless you prove them, you cannot use any results from the course notes that we have not covered.
\item You are NOT allowed to consult external sources when working on the exam.  This includes people outside of the class, other textbooks, and online resources.
\item You are NOT allowed to copy someone else's work.
\item You are NOT allowed to let someone else copy your work.
\item You are allowed to discuss the problems with each other and critique each other's work.
\end{enumerate}

The exam is due to my office by 5\textsc{pm} on \textbf{Tuesday, November 22}.  You should turn in this cover page and all of the work that you have decided to submit.

\bigskip

To convince me that you have read and understand the instructions, sign in the box below.

\bigskip

  \fbox{\parbox{7in}{
    \vspace{10pt}
    \textbf{\large Signature:} \hfill
    \vspace{10pt}
  }}

\bigskip

Good luck and have fun!

\newpage

\section*{Part 1}

Complete each of the following problems.  You should provide sufficient justification where necessary.

\begin{enumerate}

\item (3 points) Find a subgroup of $S_6$ that is isomorphic to $\mathbb{Z}_6$.

\item (3 points) Find all conjugacy classes of $D_4=\langle s,r:s^2=r^4=e, sr=r^3s\rangle$.  That is, find all of the equivalence classes of $D_4$ under the equivalence relation determined by conjugation (i.e., two elements are equivalent iff they are conjugate).

\item Let $n\in \mathbb{N}$ and define $U(n)$ to be the set of all natural numbers less than or equal to $n$ that are relatively prime to $n$.  It turns out that $U(n)$ forms a group under multiplication modulo $n$ (but you do \emph{not} need to prove this).

\begin{enumerate}

\item (3 points) Create a group multiplication table for $U(8)$.

\item (3 points) What group is $U(8)$ isomorphic to?  Be sure to justify your answer by citing recent theorems.

\end{enumerate}

\end{enumerate}

\section*{Part 2}

Let $G$ be a group and let $H\leq G$. Consider the following definitions.

\begin{definition*}
The \emph{normalizer} of $H$ in $G$ is defined via
\[
N_G(H)=\{g\in G: g^{-1}hg\in H \text{ for all } h\in H\}.
\]
\end{definition*}

\begin{definition*}
The \emph{centralizer} of $H$ in $G$ is defined via
\[
C_G(H)=\{g\in G: g^{-1}hg=h \text{ for all } h\in H\}.
\]
\end{definition*}

Now, consider the group $D_3=\langle s,r:s^2=r^3=e, sr=r^2s\rangle$ and let $H$ be the subgroup generated by the reflection $s$ and let $K$ be the subgroup generated by the rotation $r$.

\begin{enumerate}

\item (3 points) Compute all left cosets of $H$ in $D_3$.

\item (3 points) Compute all right cosets of $H$ in $D_3$.

\item (3 points) Compute all left cosets of $K$ in $D_3$.

\item (3 points) Compute all right cosets of $K$ in $D_3$.

\item (3 points) Compute the normalizer of $H$ in $D_3$.

\item (3 points) Compute the normalizer of $K$ in $D_3$.

\item (3 points) Compute the centralizer of $H$ in $D_3$.

\item (3 points) Compute the centralizer of $K$ in $D_3$.

\end{enumerate}

\section*{Part 3}

(8 points each) Prove any \textbf{4} of the following theorems.

\begin{theorem}
Let $\mathbb{R}^+$ denote the group of real numbers under addition and let $\mathbb{R}^{\times}$ denote the group of all positive real numbers under multiplication.  Then $\mathbb{R}^+$ is isomorphic to $\mathbb{R}^{\times}$.\footnote{This is Lemma 62 in the course notes.}
\end{theorem}

\begin{theorem}
Let $G$ be a group of order $pq$, where $p$ and $q$ are distinct primes.  Then $G$ has at least one element of order $p$ and at least one element of order $q$.\footnote{By symmetry, you really only need to prove this for either $p$ or $q$. Also, this is a special case of a theorem called Cauchy's Theorem, but you don't need to know this to prove the theorem.}
\end{theorem}

\begin{theorem}
Let $G$ be a group and let $H\leq G$.  Then the normalizer $N_G(H)$ is a subgroup of $G$.\footnote{See Part 2 for the definition of normalizer.}
\end{theorem}

\begin{theorem}
Let $G$ and $H$ be two groups such that $G$ is cyclic. Also, let $\phi:G\to H$ be an onto homomorphism.  Then $H$ is cyclic.
\end{theorem}

\begin{theorem}
Let $G$ be a group and fix $g\in G$. Also, define $\psi_g: G\to G$ via $\psi_g(x)=g^{-1}xg$ for all $x\in G$.  Then $\psi_g$ is an isomorphism.\footnote{An isomorphism from a group to itself is called an \emph{automorphism}.}
\end{theorem}

\begin{theorem}
Let $G$ be a group and let $H\leq G$ such that $gH=Hg$ for all $g\in G$ (i.e., all left cosets and corresponding right cosets are equal).  Then if $x\in aH$ and $y\in bH$ for $a,b\in G$, then $xy\in abH$.
\end{theorem}

\begin{theorem}
Let $G$ and $H$ be two groups and let $\phi:G\to H$ be a homomorphism.  If $g\in G$, then the order of $\phi(g)$ (in $H$) divides the order of $g$ (in $G$). 
\end{theorem}

\end{document}
